\documentclass[a4paper]{article}

%% Language and font encodings
\usepackage[english]{babel}
\usepackage[utf8x]{inputenc}
\usepackage[T1]{fontenc}
\usepackage[OT4]{polski}
\usepackage{color}

%% Sets page size and margins
\usepackage[a4paper,top=3cm,bottom=2cm,left=3cm,right=3cm,marginparwidth=1.75cm]{geometry}

%% Useful packages
\usepackage{amsmath}
\usepackage{graphicx}
\usepackage[colorinlistoftodos]{todonotes}
\usepackage[colorlinks=true, allcolors=blue]{hyperref}
\usepackage{caption}
\usepackage{array}

\title{Projekt 1 z Optymalizacji}
\author{Martyna Skiwniewska \& Klaudia Stępień }

\begin{document}
\maketitle

\section{Temat projektu}
Rozważamy następującą listę reguł wyboru zmiennych:
\begin{itemize}
\item LARGEST COEFFICIENT.
\item LARGEST INCREASE. 
\item  STEEPEST EDGE. 
\item  BLAND'S RULE.
\item  RANDOM EDGE. 
\end{itemize}
Zbadaj, jak wybór reguły wpływa na liczbę wykonywanych przez algorytm sympleks kroków.\\
Zbadaj co najmniej osiem reguł (uzupełniając powyższą listę o reguły własne, np SMALLEST INCREASE), w tym powyższe, na co najmniej dziesięciu problemach testowych.
\section{Metody}
\subsection{Metody obowiązkowe}
\begin{itemize}
\item  BLAND'S RULE MIN.  Wybór zmiennej wchodzącej o najmniejszym indeksie; jeżeli jest wiele wyborów zmiennej wychodzącej, to wybór zmiennej wychodzącej o najmniejszym indeksie. Prosta metoda bazująca na poleceniu min(self.possible\_entering()) i min(self.possible\_leaving()).
\item  LARGEST COEFFICIENT.  Wybór zmiennej o największym wspołczynniku funkcji celu. Do tej metody wykorzystałyśmy dodatkowo słownik. Trzymamy w nim pary liczb, na zerowym miejscu mamy współczynnik, a na pierwszym wartość. Dla tych par szukamy maksymalnej wartości i bierzemy jej indeks.
\item STEEPEST EDGE. Wybór zmiennej, który prowadzi do wierzchołka w kierunku najbliższym wektorowi c (gradientowi funkcji celu). Wykorzystałyśmy wzór z wykładu: 
\begin{equation*}c(x_{nowy}-x_{stary})/||x_{nowy}-x_{stary}||.
\end{equation*} 
Ponownie użyłyśmy słownik, tym razem trzymamy w nim współczynnik i odległość od gradientu funkcji celu.
\end{itemize}
\subsection{Metody dodatkowe}
\begin{itemize}
\item BLAND'S RULE MAX. Wybór zmiennej wchodzącej o największym indeksie. Analogicznie do metody BLAND'S RULE MIN tylko wykorzystujemy MAX.
\item RANDOM COEFFICIENT. Wybór losowej zmiennej. Prosta metoda bazująca na poleceniu random.choice(self.possible\_entering()) oraz random.choice(self.possible\_leaving()).
\item SMALLEST COEFFICIENT. Wybor zmiennej o najmniejszym wspolczynniku funkcji celu. Analogicznie do metody LARGEST COEFFICIENT, tylko wykorzystujemy minimalną wartość klucza dla słownika.
\item FLATTEST EDGE. Wybór zmiennej, który prowadzi do wierzchołka w kierunku najdalszym wektorowi c (gradientowi funkcji celu) - analogicznie do DEEPEST EDGE.
\item MIXED. Losowy wybór zmiennej wchodzącej sposród dostępnych metod i losowy wybór zmeinnej wychodzącej.
\end{itemize}

\section{Testy}
Poniżej przedstawiamy wyniki testów dla każdej z metod.
\vspace{0.2in}
\todo[inline, color=green!40]{T1 - AmericanSteelProblem}
\todo[inline, color=green!40]{T2 - BeerDistributionProblem}
\todo[inline, color=green!40]{T3 - ComputerPlantProblem}
\todo[inline, color=green!40]{T4 - Furniture}
\todo[inline, color=green!40]{T5 - WhiskasModel}
\todo[inline, color=green!40]{T6 - WhiskasModel2}
\todo[inline, color=green!40]{T7}
\todo[inline, color=green!40]{T8}
\todo[inline, color=green!40]{T9}
\todo[inline, color=green!40]{T10}
\vspace{0.5in}
\begin{tabular}{|l|c|c|c|c|c|c|c|c|c|c|}
Method & T1 & T2 & T3 & T4 &  T5 &  T6 &  T7 &  T8 & T9 & T10\\\hline
BLAND'S RULE MAX & 3 & 3 & 8 & 2 & 2 & 2 & 3 & 1 & 13 & 2\\
BLAND'S RULE MIN & 5 & 2 & 7 & 2 &  2 & 11 & 1 & 4 & 12 & 7\\ 
RANDOM COEFFICIENT & 4/5 & 2/3 & 5/8 & 2 & 2 & 2 & 1/3 & 1/2/4 & 8 & 8\\
LARGEST COEFFICIENT & 3 & 2 & 7 & 2 &  2 & 9 & 1 & 2 & 5 & 2\\
SMALLEST COEFFICIENT & 5 & 4 & 8 & 2 &  2 & 5 & 3 & 7 & 19 & 11\\
STEEPEST EDGE & 4 & 2 & 7 & 2 &  2 & 7 & 1 & 4 & 5 & 2\\
FLATTEST EDGE & 4 & 4 & 8 & 2 &  2 & 5 & 3 & 4 & 17 & 11\\
MIXED & 4 & 2/3/4 & 7 & 2 &  2 & 2/3/5 & 1/3 & 1/4/7 & 11 & 11
\end{tabular}
\vspace{0.05in}
\begin{center}
Table 1. Number of pivot steps.
\end{center}
\textcolor{green!40}{T1}:  -150050000.0\\
(3000.0, 2000.0, 3000.0, 4000.0, 3000.0, 3000.0, 2000.0, 0.0, 3000.0, 2000.0, 3000.0, 1000.0, 2000.0, 4000.0, 2000.0) \\
\textcolor{green!40}{T2}: -86000000\\
(700, 200, 900, 0, 0, 0, 300, 200, 1800, 0)\\
\textcolor{green!40}{T3}: -2178000000\\
(0, 0, 0, 0, 27/20, 1500, 0, 0, 0, 0, 0, 0, 1200, 0, 0, 0, 0, 0, 27/20, 1700, 1000)\\
\textcolor{green!40}{T4}: 32000000.0\\
(8.0, 16.0)\\
\textcolor{green!40}{T5}: -4800.0\\
(0.0, 60.0)\\
\textcolor{green!40}{T6}: -4800.0\\
(0.0, 0.0, 0.0, 0.0, 60.0, 0.0)\\
\textcolor{green!40}{T7}: -2711.53846154\\
(0.02692307692, 0.0, 0.0, 0.1153846154)\\
\textcolor{green!40}{T8}: 24000000\\
(0, 500, 0, 0, 700)\\
\textcolor{green!40}{T9}: 122250000.0\\
(0.0, 310.0, 0.0, 0.0, 650.0, 0.0, 240.0, 390.0, 0.0, 10.0, 0.0)\\
\textcolor{green!40}{T10}: 101000000\\
(0, 500, 0, 0, 0, 0, 0, 0, 700)

\section{Wnioski}
\vspace{0.05in}

\begin{tabular}{l|c|c|c|c|c|c|c|c|c|c|}
 & T1 & T2 & T3 & T4 &  T5 &  T6 &  T7 &  T8 & T9 & T10\\\hline
rozstęp & 2 & 2 & 1,5 & 0 & 0 & 9 & 2 & 6 & 14 & 9\\
mediana & 4 & 2,75 & 7 & 2 &  2 & 5 & 2 & 3,5 & 11,5 & 7,5\\ 
średnia arytmetyczna & 4,063 & 2,813 & 7,313 & 2 & 2 & 5,541 & 2 & 3,416 & 11,25 & 6,75\\
odchylenie standardowe & 0,776 & 0,843 & 0,594 & 0 &  0 & 3,261 & 0,926 & 1,825 & 5,148 & 4,2\\
minimum & 3 & 2 & 6,5 & 2 & 2 & 2 & 1 & 1 & 5 & 2\\
maksimum & 5 & 4 & 8 & 2 & 2 & 11 & 3 & 7 & 19 & 11\\
\end{tabular}
\vspace{0.05in}
\begin{center}
Table 2. Statistical comparison.
\end{center}


\begin{itemize}
\item W przypadku testu T4 i T5 otrzymałyśmy taką samą liczbę kroków dla wszystkiech metod,~w~pozostałych testach pojawiły się rozbieżności.
\item Najwięcej kroków, czyli 19 pojawiło się przy testowaniu testu T9 za pomocą metody Smallest Coefficient.
\item Najmniej kroków, czyli 1 pojawiło się przy testowaniu testu T7 za pomocą metody Bland's Rule Min, Largest Coefficient, Steepest Edge oraz przy testowaniu testu T8 za pomocą metody Bland's Rule Max.
\end{itemize}

\end{document}
